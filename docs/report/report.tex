\documentclass[a4paper,10pt]{article}
\usepackage{a4wide}
\usepackage[english]{babel}
\usepackage{listings}
\usepackage{color}
\definecolor{Gray}{gray}{0.95}

% This is the list style for displaying C source code:
\lstdefinestyle{code}{
    language = C,					% The language of the code snippets
    basicstyle = \small\ttfamily,	% Font of the text
    numbers = left,					% Position of the line numbers
    numberstyle = \footnotesize,	% Style of line numbers
    frame = tb,				        % Style of surrounding frame
    framextopmargin=.75mm,         % Space margin top
    framexbottommargin=.75mm,      % Space margin bottom
    framexleftmargin=2mm,          % Space margin left
    framexrightmargin=2mm,         % Space margin right
    tabsize = 3,					% Size of tab character
    breaklines = true,             % Wrap lines of code that are too long
    columns = fullflexible,			
    showstringspaces = false,
    backgroundcolor = \color{Gray}
}

% This is the list style for displaying input/output. It is different from the style above, since you don't need C keywords to be highlighted in these listings. Line numbers are also emitted in this style.
\lstdefinestyle{stdio}{
    basicstyle = \small\ttfamily,	% Font of the text
    frame = tb,				        % Style of the surrounding frame
    framextopmargin=.75mm,         % Space margin top
    framexbottommargin=.75mm,      % Space margin bottom
    framexleftmargin=2mm,          % Space margin left
    framexrightmargin=2mm,         % Space margin right
    tabsize = 3,					% Size of tab character
    breaklines = true,             % Wrap lines of text that are too long
    columns = flexible,			
    showstringspaces = false,
    backgroundcolor = \color{Gray}
}

\title{A Genetic Algorithm for the Partition Problem}
\author{H.S. de Hoop \& N. Trigonis\\
        H.S.de.Hoop@student.rug.nl \& N.Trigonis@student.rug.nl}
%------------------------------------------------------------%
% This is where your document starts:

\begin{document}
\maketitle

\section{Problem description}
Given a multiset $S$ of $n$ integers, divide the values into $k$ parts as evenly as possible.
This is the multiway number partitioning problem, also known as the identical-machines scheduling problem.
\section{Problem analysis}
As has been communicated there is no way to solve the partition problem deterministically as a polynomial function of complexity. This means that for a large input, it becomes practically impossible to produce an output.
As a result, a probabilistic method is preferred.


We interpret the question as follows: Two line segments intersect if they have precisely one common point that is not an end-point of any of the segments.
This interpretation is motivated by the following statements:
\begin{itemize}
\item Two line segments intersect if and only if $A$ and $B$ lay on different sides of the line segment trough $P$ and $Q$. Moreover, $P$ and $Q$ lay on different sides of the line segment through $A$ and $B$.
\item All cases in which multiple points are on both line segments (i.e. overlapping segments) are categorized as not intersecting.
\end{itemize}
From the equation for a line $y = \frac{\Delta y}{\Delta x}\cdot x + c$ we conclude
$$ y\cdot \Delta x - x \cdot \Delta y - c = 0
$$ 
The advantage of the latter form is that lines parallel to the $y$-axis can be written in the same way as any other line.
For the line trough $A$ and $B$ we find: 
$\Delta x = x_B - x_A$, $\Delta y = y_B - y_A$ and $c = y_A \cdot \Delta x - x_A \cdot \Delta y$.
The line trough $A$ and $B$ splits the plane into two halves. Points on one side give a positive result and points on the other side a negative result for the following expression:
\begin{equation}\label{exprHalfPlane}
y \cdot \Delta x - x \cdot \Delta y - c
\end{equation}
Thus, two points are located in different half planes if one yields a positive and the other a negative result, when we evaluate expression (\ref{exprHalfPlane}) for both points.
We can simplify this conclusion: \\[2mm]
\centerline{\it two points are in different half planes if the product of the expressions (\ref{exprHalfPlane}) is negative.}

\section{Design}
First the points $A$, $B$, $P$ and $Q$ are read from the standard input and stored (in the variables {\tt xA}, {\tt yA}, {\tt xB}, {\tt yB}, {\tt xP}, {\tt yP}, {\tt xQ}, and {\tt yQ}). 
Next, the parameters of the line through $A$ and $B$ are computed ({\tt deltax}, {\tt deltay}, and {\tt c}). Using these, we determine whether $P$ and $Q$ are split up by the line trough $A$ and $B$ (stored in {\tt PQdivided}).


After this computation, we perform the same for the other line segment through $P$ and $Q$.
Note that the variables {\tt deltax}, {\tt deltay} and {\tt c} were used to compute {\tt PQdivided}, and can now be reused in the computation of {\tt ABdivided}, which denotes 
whether the points $A$ and $B$ are split up by the line trough $P$ and $Q$.

Finally the answer is computed from the variables  {\tt PQdivided} and {\tt ABdivided}.
It is presented to the user by printing it on the standard output.

\section{Program code}
\begin{lstlisting}[style = code, title = intersect.c]
#include <stdio.h>
#include <stdlib.h>

/* (C) Arnold Meijster & Doina Bucur, september 2015:
*/

int main(int argc, char *argv[]) {
    int PQdivided;         /* Are P and Q divided through AB? */
    int ABdivided;         /* Are A and B divided through PQ? */
    int xA, yA;            /* coordinates of point A          */
    int xB, yB;            /* coordinates of point B          */
    int xP, yP;            /* coordinates of point P          */
    int xQ, yQ;            /* coordinates of point Q          */
    int deltax, deltay, c; /* parameters of the equation      */

    /* input coordinates */
    printf("Please enter the x- and y-coordinates, separated by a space.\n");
    printf("Point A: ");
    scanf("%d %d", &xA, &yA);

    printf("Point B: ");
    scanf("%d %d", &xB, &yB);

    printf("Point P: ");
    scanf("%d %d", &xP, &yP);

    printf("Point Q: ");
    scanf("%d %d", &xQ, &yQ);

    /* determine whether P and Q are separated by the line trough A and B */
    deltax = xB - xA;
    deltay = yB - yA;
    c = yA*deltax - xA*deltay;
    PQdivided = ((yP*deltax - xP*deltay - c)*(yQ*deltax - xQ*deltay - c) < 0 );
    /* determine whether A and B are sperated by the line trough P and Q */
    deltax = xQ - xP;
    deltay = yQ - yP;
    c = yP*deltax - xP*deltay;
    ABdivided = ((yA*deltax - xA*deltay - c)*(yB*deltax - xB*deltay - c) < 0 ) + 
                 ((yA*deltax - xA*deltay - c)*(yB*deltax - xB*deltay - c) < 0 );
    /* print result */
    if (PQdivided && ABdivided) {
        printf("The line segments intersect.\n");
    } else {
        printf("The line segments do not intersect.\n");
    }
    return 0;
}
\end{lstlisting}

\section{Test results}

\begin{itemize}

\item Input: (normal case with one common point)
\begin{lstlisting}[style = stdio]
0 0
9 9
4 9
8 1
\end{lstlisting}

  Output:
\begin{lstlisting}[style = stdio]
The lines intersects each other.
\end{lstlisting}


\item Input: (normal case without common point)
\begin{lstlisting}[style = stdio]
0 0
4 8
9 1
5 8
\end{lstlisting}

  Output:
\begin{lstlisting}[style = stdio]
The lines do not intersects each other.
\end{lstlisting}


\item Input: (P on line of AB)
\begin{lstlisting}[style = stdio]
0 0
9 9
5 5
9 1
\end{lstlisting}

  Output:
\begin{lstlisting}[style = stdio]
The lines do not intersects each other.
\end{lstlisting}


\item Input: (all points on one line)
\begin{lstlisting}[style = stdio]
0 0
7 7
5 5
9 9
\end{lstlisting}

  Output:
\begin{lstlisting}[style = stdio]
The lines do not intersects each other.
\end{lstlisting}


\item Input (lines with equal start-point and end-point)
\begin{lstlisting}[style = stdio]
5 5
5 5
8 2
1 9 
\end{lstlisting}

  Output:
\begin{lstlisting}[style = stdio]
The lines do not intersects each other.
\end{lstlisting}

\end{itemize}

\section{Evaluation}
Due to our relative simple definition of line segment intersection, the program was rather easy to design and write. For our definition of intersection (i.e. only one common point), the program produces correct results. If you would change the definition to 
at least one common point, further analysis would be necessary. Then the three different results of the product: positive, negative and zero have to be investigated. 
Which is a bit harder if both expressions return zero as a result. But luckily our definition is sufficient for this assignment.

We have chosen a geometric approach. A mathematical approach would have been possible as well. Then we would have computed the intersection point given the line equations. In this case you have to take care of multiple intersection points. Moreover, you also need to test whether the calculated intersection point is on the line segments. We think that the geometric approach is a lot easier.
\end{document}
